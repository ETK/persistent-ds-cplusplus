\chapter{Introduction}
Persistence is the topic dealing with the general availability of previous
versions of a data structure for various purposes. The antonym to ``persistent''
is in this context called ``ephemeral'', which means that no previous version of
a data structure is available. In the 1989 article by Driscoll et al.
\cite{Driscoll198986}, it is described how a data structure may be made
persistent in two different ways: partially or fully.

With partially persistent data structures, any previous version is accessible,
but not modifiable, and as such changes to the data structure can only be made
on the most recent version.

With fully persistent data structures, it is possible to make changes to a
previous version of the data structure, thus creating an alternative branch in
the `history' of the data structure.

Partially persistent data structures can be compared with ``rollback'' databases
as defined in \cite{10.1109/AFIPS.1987.11}. ``Rollback'' databases allow reading
from any previous version of the data structure, but only allow changing the
most recent version.
Driscoll et al. describe a method for making any bounded in-degree data
structure partially persistent with $O(1)$ amortized space and time overhead per
version and $O(1)$ overhead for access of any version and operations on the most
recent version. This method is called Node Copying.

My thesis will discuss Node Copying versus Rollback as two different approaches
to obtaining partial persistence, using a doubly linked list as an example data
structure.

\section{Chapter overview}
In the \nameref{chapter:method} chapter, the theory behind the two approaches is
analysed, and possible ways to of optimizing the performance of the Rollback
approach are investigated. Time and space complexities are shown for retrieving
a desired version and for adding a new version by changing the most recent one.

In the \nameref{chapter:empirical-analysis} chapter, experiments run with
implementations of the approaches are described, and findings based on their
experiment data are discussed and explained. The experiments cover two different
usage scenarios. The purpose of the empirical analysis is to form a practical
foundation on which to recommend one or the other approach when partial
persistence is needed in practice.

In the \nameref{chapter:conclusion} chapter, overall conclusions and
recommendations are made based on the theoretical and empirical analyses, and
suggestions are given for future work.

\section{Scope}

In my thesis, only data structures with bounded in-degree will be considered,
and the scope is limited to partially persistent data structures.

The thesis is mainly concerned with making a doubly linked list partially
persistent.

When discussing time and space complexity, the RAM model will be used, and
big-O notation is used when analysing and comparing such complexities.
