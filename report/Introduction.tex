\chapter{Introduction}
\section{Persistence}
Persistence is the topic covering the general availability of previous versions
of a data structure for various purposes. The antonym to ``persistent'' is in
this context called ``ephemeral'', which means that no previous version of a
data structure is available. In the 1989 article by Driscoll et al.
\cite{Driscoll198986}, it is described how a data structure may be made
persistent in two different ways: partially or fully.

With partially persistent data structures, any previous version is accessible,
but not modifiable, and as such changes to the data structure can only be made
on the most recent version.

With fully persistent data structures, it is possible to make changes to a
previous version of the data structure, thus creating an alternative branch in
the `history' of the data structure.

Partially persistent data structures can be compared with ``rollback'' databases
as defined in \cite{10.1109/AFIPS.1987.11}.

In my thesis, only partially persistent data structures will be considered.

Tsotras and Kangelaris \cite{Tsotras1995237} define as the database state $s(t)$
the collection of objects that exist at t in the real-world system modeled by
the database. The ability to access $s(t)$ is here referred to as ``temporal
access''. In my comparison of different approaches in practice, the cost of
retrieving $s(t)$ will be the metric by which the approaches are measured.

I will cover the following two approaches to partial persistence:
\begin{description}
  \item[Rollback] which requires functional extensions of the underlying data
  structure in order to record the necessary information about the context in
  which an operation is carried out in order to be able to reproduce or revert
  it at a later time; and
  \item[Node copying] which requires structural extensions of the underlying
  data structure in order to store modification records and other necessary
  information within the nodes of the data structure.
\end{description}

\subsection{Rollback}

In the introduction section of \cite{Tsotras1995237} are described two basic,
na\"ive solutions to being able to return $s(t)$, ``copy'' and ``log'':
\begin{description}
  \item[copy] works by storing a full copy of the data structure after each
  operation, thus allowing one to access any version simply by using the copy
  for the desired version. The obvious drawbacks of this approach is that space
  consumption can grow very fast. E.g. if only insertions are made, the cost is
  combinatorial $O\left(\binom{n+1}{2}\right)$ since every insertion will cost
  $O(1)$ more than the previous. On the other hand, access takes just $O(1)$.
  \item[log] works by storing a log of each operation made along with any
  information necessary to revert it. When a specific version is requested for
  access, operations can be applied or unapplied until the version is reached. 
  The obvious drawbacks for this approach is that access costs $O(n)$ in the 
  worst case, and if e.g. the initial, empty version has been accessed just 
  prior to a new operation being requested, the cost of that operation would 
  also be $O(n)$ since the whole log would have to be run through to produce 
  the latest version upon which the operation could be carried out. On the other
  hand, space consumption is $O(n)$ for $n$ operations.
\end{description}

An obvious improvement on both of the above methods is to design a compound data
structure where copies are only made of a sub-set of the versions, and the
operations log is used to recreate any versions in between. The space
consumption would then become $$O(n)+O\left(\frac{\binom{n+1}{2}}{d}\right)$$
where $d$ is the distance between each full copy. Similarly, the time required
to recreate a specific version is in the worst case
$O(1)+O\left(\frac{d}{2}\right)$.

The introduction of $d$ leads to the requirement of knowing apriori the
magnitude of the number of versions to expect. If $d$ is too small, space
consumption limits might be reached too early; if $d$ is too large, it may take
unproportionally long to recreate a desired version. A solution to this problem
is automatic scaling, i.e. increasing $d$ when space consumption is nearing a
given limit and possibly removing some datastructure copies to reflect the new
value of $d$.

For large values of $d$ it is also worth considering the implementation details
of recreating the desired version using the operations log based on the
nearest full copy. If, for example, $d$ is 200, it would in the worst case be
necessary to work through 100 operations to reach the version which is exactly
in the middle of two full copies. Depending on the underlying data structure,
and the type and nature of the 100 operations, it can pay off to analyse,
prune and reorder the sequence of operations to reduce the number of I/O calls.
If, for example, the first 50 operations each insert an element, but the next 50
operations are deletions of those very same 50 elements, the entire sequence can
be reduced to a no-op, and the full copy can be presented as the answer to the
access query.

\subsection{Node copying}
Node copying is described in \cite{Driscoll198986} and works by expanding the
existing data structure node. An array of field modifications of at least $p$ is
introduced which represents any modifications made to the node after its
creation. An array of so-called ``back pointers'' is maintained in node $x$
which contains a pointer to each node $y$ that contains a normal pointer to $x$.
If a modification is requested at a time when the modifications array is full, a
copy of the node is created with the most recent values for each field stored in
the original fields, and with an empty modifictations array. The back pointers
are inherited from the original node, and each node $y_i$ which back pointer
$x_i$ refers to is updated to point to the node copy. This can cause back
propagation, but the amortized cost has a constant upper bound.
\todo[inline]{Citation needed for back propagation cost amortization.}
