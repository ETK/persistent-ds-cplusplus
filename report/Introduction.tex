\chapter{Introduction}
Persistence is the topic dealing with the general availability of previous
versions of a data structure for various purposes. The antonym to ``persistent''
is in this context called ``ephemeral'', which means that no previous version of
a data structure is available. In the 1989 article by Driscoll et al.
\cite{Driscoll198986}, it is described how a data structure may be made
persistent in two different ways: partially or fully.

With partially persistent data structures, any previous version is accessible,
but not modifiable, and as such changes to the data structure can only be made
on the most recent version.

With fully persistent data structures, it is possible to make changes to a
previous version of the data structure, thus creating an alternative branch in
the `history' of the data structure.

Partially persistent data structures can be compared with ``rollback'' databases
as defined in \cite{10.1109/AFIPS.1987.11}. Driscoll et al. describe a method
for making any bounded in-degree data structure partially persistent with $O(1)$
amortized space and time overhead per version and $O(1)$ overhead for access of
any version and operations on the most recent version. This method is called
Node Copying.

\section{Scope}

In my thesis, only data structures with bounded in-degree will be considered,
and I limit the scope to partially persistent data structures.
