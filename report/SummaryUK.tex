\chapter{Summary}

The goal of the thesis is to theoretically analyse and compare Node Copying and
Rollback, two approaches to making a data structure partially persistent
(allowing the access of any previous version), and to evaluate in practice which
of them is more suited for use in two different usage scenarios: a sequential
one, where operations are applied in long sequences generating versions with
large data structures; and a randomized one, where operations are executed in a
random order. A doubly linked list is used as the example data structure.

It is found that Node Copying performs significantly better than Rollback in
terms of space complexity regardless of the usage scenario. It also performs
better in accessing a desired version in the randomized scenario independently
of the data set size.

In the sequential scenario, the smaller time complexity related to navigating
the data structure at a given version makes an optimized implementation of
Rollback faster at reaching a given index inside a desired version, when the
data set is large enough.
