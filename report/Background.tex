\chapter{Background}

Tsotras and Kangelaris \cite{Tsotras1995237} define as the database state $s(t)$
the collection of objects that exist at t in the real-world system modeled by
the database. The ability to access $s(t)$ is here referred to as ``temporal
access''. In my comparison of different approaches in practice, the cost of
retrieving $s(t)$ will be the metric by which the approaches are measured.

I will cover the following two approaches to partial persistence:
\begin{description}
  \item[Rollback] which requires functional extensions of the underlying data
  structure in order to record the necessary information about the context in
  which an operation is carried out in order to be able to reproduce or revert
  it at a later time; and
  \item[Node copying] which requires structural extensions of the underlying
  data structure in order to store modification records and other necessary
  information within the nodes of the data structure.
\end{description}
