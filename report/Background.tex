\chapter{Background}

\section{Persistence}
In \cite{Driscoll198986}, a number of methods for making data structures
persistent are described. The one with the best bounds for partial persistence
is Node copying, which allows for $O(1)$ overhead on insertions, modifications,
deletions and access operations with $O(1)$ space overhead for any
constant bounded in-degree data structure, e.g. a list or a binary search tree.

An alternative method to modifying the data structure design itself as is done
with Node copying, one can employ the simple ``rollback'' technique. With every
modifying operation performed, a record is inserted in an operation sequence
containing enough information to roll back the operation. When a specific
version of the data structure is desired, one can roll back or forward in the
operation sequence until the version is reached. The introduction of snapshots
at known intervals can decrease the time cost for accessing a specific version
by allowing one to first jump to the snapshot nearest the desired version, then
rolling back or forward until the version is reached. The interval between
snapshots introduces a parameter with which one can tune the trade-off between
access time and space consumption.

\subsection{Node copying}
Node copying is described in \cite{Driscoll198986} and works by expanding the
existing data structure node. An array of field modifications of at least $p$ is
introduced which represents any modifications made to the node after its
creation. An array of so-called ``back pointers'' is maintained in node $x$
which contains a pointer to each node $y$ that contains a normal pointer to $x$.
If a modification is requested at a time when the modifications array is full, a
copy of the node is created with the most recent values for each field stored in
the original fields, and with an empty modifictations array. The back pointers
are inherited from the original node, and each node $y_i$ which back pointer
$x_i$ refers to is updated to point to the node copy. This can cause back
propagation, but the amortized cost has a constant upper bound.
\todo[inline]{Citation needed for back propagation cost amortization.}

\subsection{Rollback}
\subsubsection{Operation sequence optimization}
When observing the following, simple sequence of operations between version
$v_i$ and $v_4$
