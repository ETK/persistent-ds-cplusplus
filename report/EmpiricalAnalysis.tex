\chapter{Empirical Analysis}

\section{Implementation}


\subsection{Execution environment}
The empirical analysis is based on the output of executing different
implementations on a machine with the following specifications:

% \begin{table}[!ht]
\begin{tabular}{|l|l|}
\hline
CPU & Intel\textregistered Core\texttrademark i5-2400 CPU @ 3.10GHz $\times$ 4
\\
\hline
Memory & Hynix/Hyundai 2048 MB DDR3 RAM @ 1333 MHz $\times$ 2 \\
\hline
Operating System & Ubuntu 12.10 64-bit \\
\hline
\end{tabular}
%\end{table}

The machine has been put into single-user mode prior to execution, and no other
non-operating system processes have run at the same time apart from the login
shell.

\subsection{Implemented operations}

I have implemented Node Copying and Rollback for a doubly linked list. The
Rollback implementation exists in two variants:

\begin{description}

  \item[Blackbox] Uses a simple, ephemeral doubly linked list for applying each
   operation between full copy and destination version separately.
  
  \item[Eliminate-Reorder] Eliminates superfluous operations using Algorithm
  \ref{alg:eliminate-ops} and then reorders the remaining ones using Algorithm
  \ref{alg:opsort} before applying them in a single iteration through the
  underlying ephemeral doubly linked list.

\end{description}

Both variants default to a maximum of 4000 full copies and an initial distance
of 65 operations between them. Whenever 4000 full copies have been made, every
second one of them is deleted (i.e. the number of full copies then becomes
2000), and the distance is doubled.

All implementations support the following operations:

\begin{description}

  \item[$\textsc{insert}(i,d)$] Inserts an element with data $d$ at index $i$.

  \item[$\textsc{modify}(i,d)$] Modifies the data of the element at index $i$ to
  $d$.

  \item[$\textsc{remove}(i)$] Removes the element at index $i$.

\end{description}

These first three represent the usual operations available on a linked list,
with bulk operation friendly parameters (conventional linked list
implementations take a pointer to a node instead of an index). The difference is
that they create a new version of the data structure.

\begin{description}

  \item[$\textsc{head}(v)$] Returns the head node of the list at version
  $v$.

  \item[$\textsc{size}(v)$] Returns the size of the list at version $v$.

\end{description}

These next two are also usually available in a linked list implementation, but
these variants take the number of a version $v$ from which to return the
information.

\begin{description}

  \item[$\textsc{access}(v,i)$] Returns the value of the element at index $i$ in
  version $v$.

  \item[$\textsc{num\_versions}()$] Returns the total number of versions.

\end{description}

These last two are implemented for the convenience of testing bulk usage of the
data structure.

Let the operations $\textsc{insert}(i,d)$, $\textsc{modify}(i,d)$,
$\textsc{remove}(i)$ and $\textsc{access}(v,i)$ be the ones benchmarked and
henceforth be referred to as ``the operations''.

\subsection{Implemented usage scenarios}

In benchmarking the performance of the approaches in practice, we will look at
how they perform under various usage scenarios. We cosider only scenarios
concerning at least a sizeable total number of operations (i.e. $N \ge 1000$).

When either of the approaches are to be used in practice, one can imagine
different usage scenarios:

\begin{description}

  \item[Random] The operations are executed in random order with no particular
  pattern, except that if an operation is illegal (such as removing when the
  list is empty), an \textsc{insert} operation is chosen. They may be weighted
  such that there is different probability for choosing different operations.
  
  In my experiments, the probability is equal (25\%) for \textsc{insert},
  \textsc{modify}, \textsc{remove} and \textsc{access} operations. Thus, the
  number of operations of each type will be close to $\frac{1}{4}$ of the total
  number of operations.

  \item[Sequential] The different types of operations are executed in sequences.
  
  In my experiments, equally many \textsc{insert}, \textsc{modify},
  \textsc{remove} and \textsc{access} operations are made, and in that order.
  Thus, the number of operations of each type will be exactly a quarter of the
  total number of operations. The index on which the operations operate on are
  still randomly chosen unless otherwise specified.

\end{description}


\subsection{Program executable}

The above scenarios have been implemented in the program \texttt{msc}, which
accepts the following arguments:

\begin{description}

  \item[\texttt{-\@{}-count}/\texttt{-c \{num\}}] Total number of operations to
  carry out (default: 1000).

  \item[\texttt{-\@{}-randomize-operations}] If passed, applies the operations
  chosen at random between the four types. Otherwise, insertions are applied
  first, then modifications, then removals, then access operations (default:
  off).

  \item[\texttt{-\@{}-rollback-eliminate-reorder}/\texttt{-l}] Will use the
  Elimination-Reorder Rollback implementation (default).

  \item[\texttt{-\@{}-rollback-blackbox}/\texttt{-r}] Will use the Blackbox
  Rollback implementation.

  \item[\texttt{-\@{}-node-copying}/\texttt{-p}] Will use the Node Copying
  implementation.

  \item[\texttt{-\@{}-max-snapshot-dist}/\texttt{-d \{num\}}] Maximum number of
  operations between full copies (default: 65, applies only to the Rollback
  implementations).

  \item[\texttt{-\@{}-max-num-snapshots}/\texttt{-m \{num\}}] Maximum number of
  full copies before adaptive fallback is carried out (default: 4000, applies
  only to the Rollback implementations).

  \item[\texttt{-\@{}-head-only}/\texttt{-h}] If set, all operations will work
  on index 0, i.e. the head of the list. This also includes \textsc{access}
  operations.

  \item[\texttt{-\@{}-store-results}/\texttt{-s}] If set, will store results in
  an SQLite database file ``sqlite.db''.
  
\end{description}

\section{Time Results}
\label{sec:time-results}

I have run a series of experiments with various combinations of program
arguments in order to determine the time-related performance of each
implementation. The same arguments are used 10 times for each of the following
total operation counts, which are exponentially spaced between 1000 and 2000000:

\begin{tabular}{|l|l|l|l|l|l|l|l|l|l|}
\hline
1000 & 2327 & 5415 & 12599 & 29317 & 68219 & 158740 & 369375 & 859506 & 2000000\\
\hline
\end{tabular}

Please note that experiments are not run for the last three counts, since with
the Rollback implementations they would exceed the available memory of the
execution environment, and with the Node Copying implementation they would take
exceedingly long to complete.

If the \texttt{-\@{}-head-only} argument is given, the index passed to the
operations is 0. Otherwise, the index is randomly selected from the range
$[0..N[$ where $N$ is the number of elements in the list. Effectively, the
Elimination-Reorder implementation should perform worse than the Blackbox when
this argument is given, since there is no reordering to be done when all
operations work on index 0.

For \textsc{access} operations, the version is randomly chosen between the
versions which exist when the operation is run.

%\subsection{Graphs}

In the following graphs, all data points are averages over 10 runs with
identical parameters, each representing the duration spent on the respective
operations. The Y error bars indicate $\pm 1.96$ times the standard deviation,
i.e. a 95\% confidence interval. Data points for the Sequential scenario are
marked by boxes, while those for the Random scenario are marked by diamonds.

It is noted in the captions if the \texttt{-\@{}-head-only} argument is given.

The following findings are based on inspecting the results of the experiments by
analyzing the graphs and the underlying data. Please note that, unless otherwise
stated, the count axis indicates the count of the operation type being
discussed, and \emph{not} the total operation count.

\subsection{Access}
For this operation, the time measured it that which it takes to get the head of
the list of a random version and then iterating to a randomly chosen index
within that list.

\subsubsection{Random scenario}
The following conclusions regarding the Random scenario are evident when
examining the diamond-shaped data points plotted in Figure
\ref{fig:access-duration-per-op-vs-count.pdf}:

\begin{itemize}

  \item Node Copying is faster than either of the Rollback implementations
  regardless of \texttt{count}.
  
  This is most likely due to the constants being lower for Node Copying than for
  Rollback when accessing the head of a version, and the cost eventually
  increasing for the Rollback variants.
  
  \item The Blackbox variant of the Rollback implementation shows to be faster
  than the Elimination-Reorder variant.
  
  This is most likely due to the list never growing very large, owing to the
  \textsc{remove} operations occurring with 25\% probability mixed between other
  operations, shrinking the list. When the list is not very long, it is cheaper
  to iterate the list from the head to carry out the operations rather than
  running $O(n^2)$ time algorithms on the operation sequences.
  
  \item The Rollback implementations both show an trend to increase toward the
  higher end of the count axis, which can be explained by the fact that the
  distance between full copies will double after operation no.
  $4000\times65 = 2.6\times10^5$ and again after operation no.
  $2.6\times10^5+130\times2000=5.2\times10^5$ changing operations.
  
  When these events occur, some time is spent on discarding full copies, and
  afterwards it will take twice as long in expectation to reach a uniformly
  randomly chosen version.

\end{itemize}

\begin{figure}[!htb]
  \center
  \includegraphics[width=0.8\textwidth]%
  {figures/graphs/compare_rollbacks/access-duration-per-op-vs-count.pdf}

  \caption{Results for \textsc{access} operations.}

  \label{fig:access-duration-per-op-vs-count.pdf}
\end{figure}

\subsubsection{Sequential scenario}

The following conclusions are evident when examining the box-shaped data points
plotted in Figure \ref{fig:access-duration-per-op-vs-count.pdf}. Please note
that prior to the \texttt{count} \textsc{access} operations, equally many
\textsc{insert}, \textsc{modify} and \textsc{remove} operations have been
executed, and thus a total of $1+\frac{3}{4}$\texttt{count} versions exist.

\begin{itemize}

  \item Up to total operation counts of about 5415 corresponding to approx.
  $\frac{3}{4}5415 \approx 4061$ versions, Node Copying is faster than Rollback.
  But since longer sequences of \textsc{insert} operations are carried out with
  higher total operation counts, the average length of the list at a randomly
  chosen version increases.
  
  Once the head of a given version has been obtained, Node Copying is slower at
  iterating through the list than Rollback. With large enough data sets, the
  list will be so long that the cost of iterating the list becomes greater, when
  compared to Rollback, than the benefit of being faster at getting the head
  node.
  
  If the Blackbox variant of Rollback could allocate enough memory to complete a
  greater number of operations, it would likely surpass Node Copying in speed
  per \textsc{access} operation around $10^5$.

  \item The time spent on optimizing the operations sequence in the
  Elimination-Reorder variant makes it perform worse at \textsc{access}
  operations than the Blackbox variant with low operation counts.
  
  Once enough elements are inserted, it pays off to reorder the operations for a
  single iteration through the list instead of $n$ potentially full iterations
  for $n$ operations.
  
  This is the case already at a total operations count of 2327, which
  corresponds to a list length of at least 582 elements, where
  Elimination-Reorder becomes faster than Blackbox. With slightly more than 5415
  total operations, corresponding to a list length of at least 1354, it is also
  faster than Node Copying.

\end{itemize}

\subsection{Insert}
For this operation, the time which is measured is that which it takes to insert
an element at a random index of the list in its most recent version.

\subsubsection{Random scenario}
The following conclusions regarding the Random scenario are evident when
examining the diamond-shaped data points plotted in Figure
\ref{fig:insert-duration-per-op-vs-count.pdf}:

\begin{itemize}

  \item The Blackbox variant of Rollback dominates the other implementations
  with any operation count. Elimination-Reorder starts out the slowest, but
  after 5415 total operations, corresponding to at least 1354 \textsc{insert}
  operations, it becomes faster than Node Copying. Eventually,
  Elimination-Reorder and Blackbox are very nearly equally fast.
  
  The simple explanation to this result is that Node Copying is slower than the
  Rollback implementations at iterating through the list to find the point of
  insertion. Once enough versions exist with long enough lists, this deficiency
  will cost Node Copying more than it gains from its fast retrieval of the head
  node.

\end{itemize}

\begin{figure}[!htbp]
  \center
  \includegraphics[width=0.8\textwidth]%
  {figures/graphs/compare_rollbacks/insert-duration-per-op-vs-count.pdf}

  \caption{Results for \textsc{insert} operations.}

  \label{fig:insert-duration-per-op-vs-count.pdf}
\end{figure}

\subsubsection{Sequential scenario}
The following conclusions are evident when examining the box-shaped data points
plotted in Figure \ref{fig:insert-duration-per-op-vs-count.pdf}. Please note
that prior to the \texttt{count} \textsc{access} operations, equally many
\textsc{insert}, \textsc{modify} and \textsc{remove} operations have been
executed, and thus a total of $1+\frac{3}{4}$\texttt{count} versions exist.

\begin{itemize}
  
  \item The Rollback variants are virtually equally fast. This is because, in
  contrast to the Random scenario, the most recent version is already available
  when the \textsc{insert} operation is to be applied, and thus that version
  does not need to be retrieved before the operation can be applied.
  
  \item Node Copying is slower than both Rollback variants, and increasingly
  more so as more \textsc{insert} operations are carried out.
  
  This result shows most clearly how, as pointed out before, Node Copying is
  slower at iterating through the list to the point where the changing operation
  is supposed to take place.
\end{itemize}

\subsubsection{Other operations}
The \textsc{modify} and \textsc{remove} operations show virtually the same
results as the \textsc{insert} operation, since the outcomes mostly depend on
how long the list gets in the versions created --- a figure which depends
primarily on the number of \textsc{insert} operations.

The graphs are included in Figure
\ref{fig:modify-remove-duration-per-op-vs-count.pdf} nevertheless for
comparison.

\begin{figure}[!htbp]
  \center
  \includegraphics[width=0.8\textwidth]%
  {figures/graphs/compare_rollbacks/modify-duration-per-op-vs-count.pdf}
  \includegraphics[width=0.8\textwidth]%
  {figures/graphs/compare_rollbacks/remove-duration-per-op-vs-count.pdf}

  \caption{Results for \textsc{modify} and \textsc{remove} operations.}

  \label{fig:modify-remove-duration-per-op-vs-count.pdf}
\end{figure}

\subsubsection{Total duration}
When looking at the total duration, i.e. the time from start to finish of the
entire scenario, it turns out that Node Copying is the fastest --- see Figure
\ref{fig:total-remove-duration-per-op-vs-count.pdf}.

\begin{figure}[!htbp]
  \center
  \includegraphics[width=0.8\textwidth]%
  {figures/graphs/total-duration-vs-count-avg.pdf}
  \includegraphics[width=0.8\textwidth]%
  {figures/graphs/total-duration-per-op-vs-count-avg.pdf}

  \caption{Total duration and duration per operation across all the operations
  of one experiment for each batch size.}

  \label{fig:total-remove-duration-per-op-vs-count.pdf}
\end{figure}

\section{Space results}
In order to estimate the memory usage of the different implementations,
preprocessor directives have been introduced which control whether time or
measurements are made. If the \texttt{MEASURE\_SPACE} symbol is defined, no
lines of code which measure the time of operations are compiled. Instead, code
lines are introduced which estimate the memory usage.

The memory usage for an instance of the Rollback implementation with $F$ full
copies and $N$ total operations is estimated according to the following formula:
\begin{eqnarray*}
\textnormal{total\_space} &=& \textsc{size\_of}(\textnormal{ephemeral node})
\times
\sum_{i=1}^F {\textsc{size}(\textnormal{full copy}_i)}\\
&&+
\textsc{size\_of}(\textnormal{operation record})\times N\\
&&+ \textsc{size\_of}(\textnormal{full copy record}) \times F\\
&&+ \textsc{size\_of}(\textnormal{auxillary DS})
\end{eqnarray*}

Please note that the total memory reserved by the program when using
Elimination-Reorder for large data sets is measurably smaller than when using
Blackbox. This is because fewer nodes are allocated which would be deleted again
as part of getting form version $v_current$ to version $v_x$.

For Node Copying, the estimation is more accurate, given that every time a new
persistent node is created, either due to an \textsc{insert} operation or due to
a copy being made as described in Section \ref{subsec:node-structure-expansion},
a counter is incremented by the size of a persistent node.

In Figure \ref{fig:space-results.pdf} it is clearly visible that Rollback uses
significantly more space than Node Copying in the Sequential scenario -- and
consistently more so in the Random scenario.

\begin{figure}[!ht]
  \center \includegraphics[width=0.8\textwidth]%
  {figures/graphs/space-results.pdf}

  \caption{Estimated memory usage.}

  \label{fig:space-results.pdf}
\end{figure}
