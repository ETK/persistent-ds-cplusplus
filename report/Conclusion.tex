\chapter{Conclusion}
\label{chapter:conclusion}

\section{Recommendations}
If sufficient programming time is available for implementing an optimized
Rollback approach, it is the recommended option when dealing with large numbers
of operations in a manner similar to the Sequential scenario --- provided that
sufficient system memory is available. This is especially the case if many
\textsc{access} operations are expected to be made which are followed by
navigation far into the produced version.

If on the other hand a data structure is to be made partially persistent, which
is not suitable for optimizations --- such as those which went into the
Elimination-Reorder variant of the Rollback implementation of the partially
persistent doubly linked list.

If the effective sizes of the data structure in the various versions do not
become too large, it is recommended to apply the Node Copying approach, unless
very few \textsc{access} operations are expected.

\section{Future work}
It could be investigated whether compression techinques could be applied when
storing the full copies and/or the operations log in the Rollback
implementations. If the benefit in terms of lower memory usage is great enough,
it would allow working with larger data sets than with the implementations used
in this thesis.

More advanced data structures, such as binary search trees, could be implemented
to see whether the same general conclusions apply, or if they are specific to a
doubly linked list. Notably, it would be interesting to see if eliminating
superfluous operations or optimizing the order of the operations sequence or
other such optimizations are possible and/or feasible for more advanced data
structures.

More elaborate usage scenarios could be implemented and tested using the
existing framework. E.g. version access patterns showing how well-known
algorithms would perform, such as planar point location.

It could also be investigated whether approaches from different paradigms could
effectively provide partial persistence and be compared to the existing
implementations.
