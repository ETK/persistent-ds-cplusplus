\chapter{Conclusion}
\label{chapter:conclusion}
I have analysed and compared two approaches to making a data structure partially
persistent.

The two approaches have their strengths and weaknesses. As identified in the
Method chapter and shown in the Empirical Analysis chapter, they yield different
results under different usage scenarios.

% \section{Recommendations}
If sufficient programming time is available for implementing an optimized
Rollback approach, it is the recommended option when dealing with large numbers
of operations in a manner similar to the Sequential scenario --- provided that
sufficient system memory is available. This is especially the case if many
\textsc{access} operations are expected to be made which are followed by
navigation far into the produced version.

On the other hand, a data structure which is not suitable for optimizations such
as those which went into Eliminate-Reorder, could still be made partially
persistent with reasonable effort using Node Copying, and the results would be
better than when using Blackbox Rollback.

If the effective sizes of the data structure in the various versions do not
become too large, it is recommended to apply the Node Copying approach, unless
very few \textsc{access} operations are expected.

\section{Future work}
It could be investigated whether compression techinques could be applied when
storing the full copies and/or the operations log in the Rollback
implementations. If the benefit in terms of lower memory usage is great enough,
it would allow working with larger data sets than with the implementations used
in this thesis.

More advanced data structures, such as binary search trees, could be implemented
to see whether the same general conclusions apply, or if they are specific to a
doubly linked list. Notably, it would be interesting to see if eliminating
superfluous operations or optimizing the order of the operations sequence or
other such optimizations are possible and/or feasible for more advanced data
structures.

Caching techniques or other optimizations could be employed to secure faster
navigation of the underlying data structure with Node Copying.

More elaborate usage scenarios could be implemented and tested using the
existing framework. E.g. version access patterns showing how well-known
algorithms would perform, such as planar point location. Different probabilities
in the Random and Sequential scenarios could also simulate different usage
patterns.

It could be investigated whether approaches from different paradigms could
effectively provide partial persistence and be compared to the existing
implementations.

It might also be worthwhile to find efficient ways of converting between the two
approaches such that if conditions change, the one which is most efficient can
be used.
